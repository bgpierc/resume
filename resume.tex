\documentclass[10pt]{article}
    \usepackage[english]{babel}
    \usepackage{lipsum}
    \usepackage{hyperref}
    %sty file

\usepackage{config/minimal-resume}

% custom command and packages

% Note: Put your custom commands here. Left empty intentionally.
\pagenumbering{gobble}


    \hypersetup{hidelinks,
		backref=true,
		pagebackref=true,
		hyperindex=true,
		breaklinks=true,
		colorlinks=true,%linkcolor=black,
		urlcolor=blue,
		bookmarks=true,
		bookmarksopen=false,
		pdftitle={Resume},
		pdfauthor={Benjamin Pierce}}
		
		

    \begin{document}
    
%\usepackage{times}
%\usepackage{fontspec}
%\defaultfontfeatures{Mapping=tex-text,Scale=MatchLowercase}
%\setmainfont[Path = fonts/]{Montserrat-Light}
%\setmonofont{Lucida Sans Typewriter}




\begin{center}
	% Personal
	% -----------------------------------------------------
	{\fontsize{\sizeone}{\sizeone}\fontspec[Path = fonts/]{Montserrat-Regular} Benjamin G. Pierce}
	\\
	\vspace{1mm}
	{\fontsize{1em}{1em}\fontspec[Path = fonts/]{Montserrat-Light} \href{mailto:bepierce@ucsd.edu}{bepierce@ucsd.edu} -- (614) 787-8389  -- \href{http://bgpierc.github.io}{bgpierc.github.io}}
\end{center}
% Chapter: Education
% ------------------
\chap{RESEARCH INTERESTS}{

{\fontspec[Path = fonts/]{CrimsonText-SemiBold} Photovoltaics \thinspace \thinspace  \thinspace \thinspace Modeling \thinspace \thinspace  \thinspace \thinspace Machine Learning \thinspace \thinspace  \thinspace \thinspace Energy Systems \thinspace \thinspace  \thinspace \thinspace High-Performance Computing}
}
\chap{EDUCATION}{
	
	\school
	{University of California San Diego}
	{Fall 2023 – }
	{Ph.D Mechanical Engineering}
	{La Jolla, CA}
	{\begin{newitemize}
			\item Advisor: Prof. Jan Kleissl
			%\item Funded via NSF GRFP

		\end{newitemize}
	}

	\school
	{Case Western Reserve University}
	{Aug 2017 – May 2021}
	{B.S. Computer Science; GPA: 3.6}
	{Cleveland, OH}
	{\begin{newitemize}
		\item Advisor: Prof. Roger H. French
	%	\item GPA: 3.6
       % \item Coursework: Algorithms, Databases, Machine Learning, Theoretical Computer Science, Cryptology, Linear Algebra, Probabilistic Graphical Models, High Performance Computing, Computational Perception
  %      \item Minor in Applied Data Science
		%\item Dual Degree Program- B.S./M.S. completed simultaneously
        \end{newitemize}
	}
%	{\fontspec[Path = fonts/,LetterSpace=10]{CrimsonText-SemiBold} 
%Research Interests} 

}
% Chapter: Work Experience
% ------------------------
\chap{EXPERIENCE}{
	
%	\job
%	{Sandia National Laboratories}
%	{May 2020 – Present}
%	{R\&D Intern- Photovoltaic Systems Evaluation Laboratory}	
%	{Albuquerque, NM}
%	\
%	
%	\textit{Improving Single Axis Tracking Algorithms Using Sky Imagery and Machine Learning}
%	{\begin{newitemize}
%	\item Supervisors: Joshua Stein, Jennifer Braid, Daniel Riley
%	\item Objective: Improve energy yield of single axis trackers via a new, machine learning based control algorithm that takes sky images as input
%	\item Cleaned data set of over 100K images (months of 1 minute interval data) in Sandia HPC enviroment
%	\item Created novel multi-input convolutional neural network to find angle of maximal irradiance
%	\item Model gives a mean absolute percent error of under 5\% on most days
%	\end{newitemize}}


	\job
	{Sandia National Laboratories}
	{October 2021 - Current}
	{
	 \href{https://energy.sandia.gov/programs/renewable-energy/photovoltaic-solar-energy/people/benjamin-pierce/}{Member of the Technical Staff}, R\&D Systems Engineering
%	Student Intern (2020-2021)
	}
%	{Student Intern (2020-2021)
%	R\&D Systems Research Analyst (2021-Current) }	
	{Albuquerque, NM}
	\
	%\textit{Project: Improving Single Axis Tracking Algorithms Using Sky Imagery and Machine Learning}
	{\begin{newitemize}
		%	\item Devised novel control algorithms for single axis trackers (SATs), resulted in submitted patent application
			\item Used sky images and novel sensors to forecast and nowcast cloud coverage and irradiance for SATs.
			\item Modeled PV performance for continental US for factors including terrain slope, shading, and degradation.
			\item Used Sandia High Performance Computing clusters to accelerate R\&D for multiple projects.
		%	\item Presented and published original research at professional conferences and workshops.
	%\item Supervisors: Joshua Stein, Jennifer Braid, Daniel Riley
    %\item Supported via by DOE Technology Commercialization Fund (\$500K )
	%\item Improve energy yield of single axis trackers via a new, machine learning based control algorithm that takes sky images as input
	%\item Cleaned data set of over 100K images (months of 1 minute interval data) in Sandia HPC enviroment
	%\item Created novel multi-input convolutional neural network to find angle of maximal irradiance
	%\item Model gives a mean absolute percent error of under 5\% on most days
	\end{newitemize}}
    
    
\vspace{0.25cm}

	\job
{Sandia National Laboratories}
{May 2020-October 2021}
{
%	\href{https://energy.sandia.gov/programs/renewable-energy/photovoltaic-solar-energy/people/benjamin-pierce/}{Member of the Technical Staff} (2021-Present)
	
	Student Intern
}
%	{Student Intern (2020-2021)
	%	R\&D Systems Research Analyst (2021-Current) }	
{Albuquerque, NM}
\


%\textit{Project: Improving Single Axis Tracking Algorithms Using Sky Imagery and Machine Learning}
{\begin{newitemize}
		\item Devised novel control algorithms for single axis trackers (SATs), resulted in submitted patent application and DOE funding.
%		\item Used sky images and novel sensors to forecast and nowcast cloud coverage and irradiance for SATs.
	%	\item Modeled PV performance for continental US for factors including terrain slope, shading, and degradation.
	%	\item Used Sandia High Performance Computing clusters to accelerate R\&D for multiple projects.
		%	\item Presented and published original research at professional conferences and workshops.
		%\item Supervisors: Joshua Stein, Jennifer Braid, Daniel Riley
		%\item Supported via by DOE Technology Commercialization Fund (\$500K )
		%\item Improve energy yield of single axis trackers via a new, machine learning based control algorithm that takes sky images as input
		%\item Cleaned data set of over 100K images (months of 1 minute interval data) in Sandia HPC enviroment
		%\item Created novel multi-input convolutional neural network to find angle of maximal irradiance
		%\item Model gives a mean absolute percent error of under 5\% on most days
\end{newitemize}}


\vspace{0.25cm}
	\job
	{Solar Durability and Lifetime Extension Center}
	{Aug 2018 – May 2021}
	{\href{https://engineering.case.edu/centers/sdle/ben-pierce}{Research Assistant}}
	{Cleveland, OH}
	\
	
    %Duties include data analytics, PV characterization in the lab, and computational infrastructure maintenance. 
%	\textit{Video/Image Processing on Crystal formation on Ni-Al alloys}
	{\begin{newitemize}
	%	\item{Faculty: Roger French, Masayoshi Adachi, Hiroyuki Fukuyama}
%		{\begin{newitemize}
%			\item{Collaborated on a joint project with researchers from Tohoku University, Japan}

		\item Used image processing to find the rate of crystallization of AlN on a molten Al-Ni alloy.
		\item Utilized unsupervised machine learning to sort electroluminescence images based on defects and/or damage.
		\item Investigated electrical impact of cracks on Si PV cells.
	%	\item Supported and maintained research group high performance computing infrastructure. 
	
		\end{newitemize}}
%		\item {Objective: Analyze video of rotating, molten droplet of NiAl with the aim of determining a rate of crystallization}
%		\item{Analyzed over 100,000 images across 4 samples at varying temperature and composition}
%		%\item {Found area of surface of droplet using Canny edge detection and density-based clustering}
%		%\item {Accounted for rotation of droplet and backside crystallization through sliding-window estimation} 
%		\item {Confirmed theoretical Avrami crystallization behavior with data-driven model}
%		\end{newitemize}}
%	
%	\
	
%	\textit{Project: Feature Extraction and Unsupervised Learning on Electroluminescence Images}
}	
%{\begin{newitemize}
%		\item{Faculty: Roger French, Jennifer Braid}
%		\item{Objective: Use unsupervised learning to classify types of degradation of solar modules through electroluminescence images in a dataset of 11,000 images}
	%	\item{Experiments: Took electroluminescence measurements on mini-modules and adjusted data processing step to enable further analysis}
	%	\item{Extracted local features (blots of corrosion, darkening) using algorithms such as SIFT and KAZE}
	%	\item{Found module-level features with Haralick/GLCM features and specialized extraction methods}
	%	\item{Modeled local features using bag-of-words model, and applied hierarchical clustering to identify classes}
		%\item{Parallelized operations on HPC cluster with 100Tb of disk, 180 cores, and 2Tb RAM}
		
		
	%\end{newitemize}}

\chap{PUBLICATIONS}
	{\begin{newitemize}
    
     \item{M. Adachi, S. Hamaya, D. Morikawa, {\fontspec[Path = fonts/]{CrimsonText-SemiBold} B. G. Pierce}, A. M. Karimi, Y. Yamagata, K. Tsuda, R. H. French,  H. Fukuyama, ``Temperature dependence of crystal growth behavior of AlN on Ni-Al and demonstration of thick AlN film growth using electromagnetic levitation and computer vision technique'' in Materials Science in Semiconductor Processing, 1/1/2023 [\href{https://www.sciencedirect.com/science/article/pii/S136980012200693X}{Online}] }
    
     \item{{\fontspec[Path = fonts/]{CrimsonText-SemiBold} B. G. Pierce}, J. L. Braid, J. S. Stein, and D. Riley, ``Cloud Segmentation and Motion Tracking in Sky Images,” IEEE J. Photovoltaics, 10/17/2022, [\href{https://ieeexplore.ieee.org/abstract/document/9950360/}{Online}] }
   
    \item{{\fontspec[Path = fonts/]{CrimsonText-SemiBold} B. G. Pierce}, J. L. Braid, J. S. Stein, J. Augustyn, and D. Riley, ``Solar Transposition Modeling via Deep Neural Networks With Sky Images,” IEEE J. Photovoltaics, 11/22/2021 [\href{https://ieeexplore.ieee.org/abstract/document/9623380}{Online}] }
    
     \item{C. M. Whitaker, {\fontspec[Path = fonts/]{CrimsonText-SemiBold} B. G. Pierce}, R. H. French, and J. L. Braid, ``Properties of PV Cell Fractures and Effects on Performance of Al-BSF and PERC Modules,” in IEEE 48th Photovoltaic Specialists Conference, 6/6/2021 [\href{https://ieeexplore.ieee.org/abstract/document/9519030}{Online}] }
    
    \item {{\fontspec[Path = fonts/]{CrimsonText-SemiBold} B. G. Pierce}, A. M. Karimi, J. Liu, R. H. French, and J. L. Braid, ``Identifying Degradation Modes of Photovoltaic Modules Using Unsupervised Machine Learning on Electroluminescence Images,'' in 2020 47th IEEE Photovoltaic Specialists Conference, 6/15/2020 [\href{https://ieeexplore.ieee.org/abstract/document/9301021}{Online}] }
    
    
    \item{C. M. Whitaker, {\fontspec[Path = fonts/]{CrimsonText-SemiBold} B. G. Pierce}, A. M. Karimi, R. H. French, and J. L. Braid, ``PV Cell Cracks and Impacts on Electrical Performance,'' in 47th IEEE Photovoltaic Specialists Conference, 6/15/2020 [\href{https://ieeexplore.ieee.org/abstract/document/9300374}{Online}] } 
    
	\item{A. M. Karimi, J. S. Fada, N. A. Parrilla, {\fontspec[Path = fonts/]{CrimsonText-SemiBold} B. G. Pierce}, M. Koyutürk, R. H. French, and J. L. Braid. ``Generalized and Mechanistic PV Module Performance Prediction from Computer Vision and Machine Learning on Electroluminescence Images,'' IEEE J. of Photovoltaics, 3/30/2020 
	[\href{https://ieeexplore.ieee.org/abstract/document/9050914/}{Online}]}
    
%    
%    \item{{\fontspec[Path = fonts/]{CrimsonText-SemiBold} B. G. Pierce}, ``Approaches to Sky Image Based Single Axis Tracker Algorithms,” presented at the 2022 15th PV Performance Modeling Workshop, Salt Lake City, UT. [\href{https://pvpmc.sandia.gov/resources-and-events/events/2022-pvpmc-workshop/}{Online}] }
    
    

    
    
    
    
    
    
    
    
%        \item{{\fontspec[Path = fonts/]{CrimsonText-SemiBold} B. G. Pierce}, J. L. Braid, J. S. Stein, and D. Riley, “Cloud Segmentation and Motion Tracking in Sky Images,” \textit{IEEE Journal of Photovoltaics}, Accepted 17 Oct 2022
%                }
%                
%        \item{{\fontspec[Path = fonts/]{CrimsonText-SemiBold} B. G. Pierce}, J. L. Braid, J. S. Stein, J. Augustyn, and D. Riley, “Solar Transposition Modeling via Deep Neural Networks With Sky Images,” \textit{IEEE Journal of Photovoltaics}, vol. 12, no. 1, pp. 145–151, 2021 . \href{https://ieeexplore.ieee.org/abstract/document/9623380}{https://ieeexplore.ieee.org/abstract/document/9623380}
%        }
%		\item{{\fontspec[Path = fonts/]{CrimsonText-SemiBold} Benjamin G Pierce}, Ahmad Maroof Karimi, Jiqi Liu, Roger H French, and Jennifer L Braid. "Identifying Degradation Modes  of Photovoltaic Modules Using Unsupervised Machine Learning on Electroluminescence Images" \textit{IEEE Photovoltaics Specialists Conference 2020}}
%		\item{Carolina M. Whitaker, {\fontspec[Path = fonts/]{CrimsonText-SemiBold} Benjamin G Pierce}, Ahmad Maroof Karimi, Roger H French, and Jennifer L Braid. "PV Cell Cracks and Impacts on Electrical Performance" \textit{IEEE Photovoltaics Specialists Conference 2020}}
%		\item{Ahmad Maroof Karimi, Justin S Fada, Nicholas A Parrilla, {\fontspec[Path = fonts/]{CrimsonText-SemiBold} Benjamin G Pierce}, Mehmet Koyutürk, Roger H French, and Jennifer L Braid. “Generalized and Mechanistic PV Module Performance Prediction from Computer Vision and Machine Learning on Electroluminescence Images.” \textit{IEEE Journal of Photovoltaics} }
%		%\item{{\fontspec[Path = fonts/,LetterSpace=10]{CrimsonText-SemiBold} Pierce, Benjamin}, Ahmad Maroof Karimi, Justin Fada, Nicholas Parrilla, J. L. Braid, Mehmet Koyuturk, and Roger French. 2019. “Feature Extraction/Machine Learning for Degradation Classification of Solar Modules.” Poster, CWRU SOURCE Intersections, August 2.}
%		%\item{{\fontspec[Path = fonts/]{CrimsonText-SemiBold} Benjamin Pierce}, Ahmad Karimi, Laura Wilson, Andrew Loach, Sonoko Hamaya, Justin Fada, Masayosi Adachi, Hiroyuki Fukuyama, Roger H. French, and J.L.W. Carter. 2019. “Image Processing on Crystallization Growth of Rotating and Levitated Alloys.” Poster, 2019 CWRU/Tohoku Symposium on Data Science in Life Sciences and Engineering, August 5.}
%		%\item{M. Adachi, S. Sonoko, A. Kanbara, L. G. Wilson, {\fontspec[Path = fonts/,LetterSpace=10]{CrimsonText-SemiBold}B. G. Pierce}, A. M. Karimi, R. H. French, J. L. W. Carter, H. Fukuyama, AlN growth behavior on Ni-Al liquid solutions, (2019). \href{http://www.ioffe.ru/iwumd4/invitedspeakers.html}{http://www.ioffe.ru/iwumd4/invitedspeakers.html} }
%        \item {Carolina M. Whitaker, {\fontspec[Path = fonts/]{CrimsonText-SemiBold} Benjamin G. Pierce}, Roger H. French, and Jennifer L. Braid, “Properties of PV Cell Fractures and Effects on Performance of Al-BSF and PERC Modules,” presented at the 48th PVSC, Virtual, 2021.
%        \item{{\fontspec[Path = fonts/]{CrimsonText-SemiBold} Benjamin Pierce}, Jennifer L. Braid, Joshua S. Stein, Jim Augustyn, Daniel Riley, "Solar Transposition Modeling via Deep Neural Networks with Sky Images", \textit{IEEE Journal of Photovoltaics}, submitted following invitation.  }
%        \item{A. M. Karimi, \fontspec[Path = fonts/]{CrimsonText-SemiBold} B. G. Pierce}, J. S. Fada, N. A. Parrilla, R. H. French, and J. L. Braid, PVimage: Package for PV Image Analysis and Machine Learning Modeling. 2020. Accessed: Feb. 28, 2020. [Online]. Available: \href{https://pypi.org/project/pvimage/}{https://pypi.org/project/pvimage/}
%        
%    
%        
%        }
%        
%        \item{
%        M. Adachi, S. Hamaya, D. Morikawa, {\fontspec[Path = fonts/]{CrimsonText-SemiBold} B. Pierce}, A. Karimi, Y. Yamagata, K. Tsuda, R. French,  H. Fukuyama, "Temperature dependence of crystal growth behavior of AlN on Ni-Al and demonstration of thick AlN film growth using electromagnetic levitation and computer vision technique" in Materials Science in Semiconductor Processing [Accepted, Oct 22] 
%        }
%        
        
	\end{newitemize}}




    
    
    
\chap{PRESENTATIONS}
{\begin{newitemize}
    \item {“Approaches to Sky Image Based Single Axis Tracker Algorithms,” presented at the 2022 15th PV Performance Modeling Workshop, Salt Lake City, UT. [\href{https://pvpmc.sandia.gov/resources-and-events/events/2022-pvpmc-workshop/}{Online}]}
    
    \item {
    ``Cloud Segmentation and Motion Tracking in Sky Images," presented at IEEE PVSC 2022
    
    }
    \item {
        ``Solar Transposition Modeling via Deep Neural Networks With Sky Images," presented at IEEE PVSC 2021
        
        }
\end{newitemize}}

%\chap{INTELLECTUAL PROPERTY}
%{\begin{newitemize}
%		\item ``Systems and Methods for Single-Axis Tracking via Sky Imaging and Machine Learning", Benjamin G Pierce, Joshua S Stein, Jennifer L Braid, Daniel Riley, U.S. Provisional Patent Application, Ser. No. 63/126,708, filed December 17, 2020.
%\end{newitemize}}


	
%	add that one Russia talk
%\chap{PROJECTS}
%	{\begin{newitemize}
%		\item{Web-scraping online sports databases to track and predict player growth from the NCAA to the NBA, published on \href{https://data.world/bgp12/nbancaacomparisons}{data.world}} with over 300 bookmarks
%		\item{Web marketplace with Flask frontend using MySQL backend}
%		\item{Peer-to-peer local area network IDE for Python}
%		\item{Raspberry Pi based handheld license plate identification device}
%		  
%	\end{newitemize}}


\chap{AWARDS \& HONORS} {

{\fontspec[Path = fonts/]{CrimsonText-SemiBold} 
	NSF Graduate Research Fellow} 
\hspace*{0pt}\hfill National Science Foundation

Awarded presigious NSF GRFP fellowship for graduate school funding. \hspace*{0pt}\hfill April 2023

\
	
	

{\fontspec[Path = fonts/]{CrimsonText-SemiBold} 
	IEEE PVSC 2022 Session Chair} 
\hspace*{0pt}\hfill IEEE PVSC 

Co-chair for Solar Resource and PV Forecasting,  \href{https://ieee-pvsc.org/PVSC49/program-full.php?page=program&displayday=6&pads=&start_range=&start_interval=&changing_days=yes&hide_details=}{Session II} \hspace*{0pt}\hfill June 2022

\




%{\fontspec[Path = fonts/]{CrimsonText-SemiBold} 
%SURES Scholar} 
%\hspace*{0pt}\hfill CWRU SOURCE 
%
%Awarded grant funding for summer research \hspace*{0pt}\hfill May 2019
%
%\
{\fontspec[Path = fonts/]{CrimsonText-SemiBold} 
Computer and Data Sciences Research Award} 
\hspace*{0pt}\hfill CWRU 

To the graduating senior demonstrating exceptional research potential \hspace*{0pt}\hfill May 2021

\

{\fontspec[Path = fonts/]{CrimsonText-SemiBold} 
Herbold Scholar} 
\hspace*{0pt}\hfill CWRU 

Awarded funding for Master's program at CWRU, declined for full-time position at Sandia \hspace*{0pt}\hfill May 2021

\

{\fontspec[Path = fonts/]{CrimsonText-SemiBold} 
	DOE Science Undergraduate Laboratory Internships (SULI)} 
\hspace*{0pt}\hfill Lawrence Berkeley National Lab

Offered SULI funding for Summer 2020, declined for Sandia internship \hspace*{0pt}\hfill May 2020

\

%{\fontspec[Path = fonts/]{CrimsonText-SemiBold} 
%Choose Ohio First Awardee- Data Science Cohort} 
%\hspace*{0pt}\hfill State of Ohio
%
%Recognized for STEM skills and awarded scholarship for education in data science\hspace*{0pt}\hfill Aug 2017-
%
%\

%{\fontspec[Path = fonts/]{CrimsonText-SemiBold} 
%Dean\textquotesingle{}s High Honors} 
%\hspace*{0pt}\hfill CWRU

%Made Dean\textquotesingle{}s list for academic success\hspace*{0pt}\hfill Aug 2017-

		%	\award
		%	{SURES Scholar}
		%	{May 2019}
		%	{Awarded grant funding for summer research}
		%	{CWRU SOURCE}
			
		%	\award
		%	{Choose Ohio First Awardee- Data Science Cohort}
		%	{Aug 2017}
		%	{Recognized for STEM skills and awarded scholarship for education in data science}
		%	{State of Ohio}
			
		%	\award
		%	{Dean\textquotesingle{}s High Honors}
		%	{Aug 2017-}
		%	{Made Dean\textquotesingle{}s list for academic success}
		%	{CWRU}
			
	}
\chap{TECHNOLOGIES}{

{\fontspec[Path = fonts/]{CrimsonText-SemiBold} 
Programming Languages} 
\hspace*{0pt}\hfill Python, R, Julia, C, Java, bash

{\fontspec[Path = fonts/]{CrimsonText-SemiBold} 
Libraries} 
\hspace*{0pt}\hfill PyTorch, TensorFlow, NumPy, sklearn, pandas, pvlib-python

{\fontspec[Path = fonts/]{CrimsonText-SemiBold} 
	Laboratory Equipment} 
\hspace*{0pt}\hfill Eternalsun Spire, electro/photoluminescence, Suns$V_{oc}$, breadboard electronics, etc.


{\fontspec[Path = fonts/]{CrimsonText-SemiBold} 
Databases} 
\hspace*{0pt}\hfill Hadoop2/Hbase, MySQL, MS SQL Server, SQLite


{\fontspec[Path = fonts/]{CrimsonText-SemiBold} 
Other} 
\hspace*{0pt}\hfill High-performance computing, \LaTeX
}

\chap{ACTIVITIES}{

{\fontspec[Path = fonts/]{CrimsonText-SemiBold} 
Association for Computing Machinery} 
\hspace*{0pt}\hfill Student Member, 2019

{\fontspec[Path = fonts/]{CrimsonText-SemiBold}  
	Institute of Electrical and Electronics Engineers} 
\hspace*{0pt}\hfill Student Member, 2020

%{\fontspec[Path = fonts/]{CrimsonText-SemiBold} 
%CWRU Hacker Society} 
%\hspace*{0pt}\hfill Member, 2017-Current

{\fontspec[Path = fonts/]{CrimsonText-SemiBold} 
Study Abroad} 
\hspace*{0pt}\hfill  Cape Town, South Africa, Summer 2018

{\fontspec[Path = fonts/]{CrimsonText-SemiBold} 
Volunteer Teaching Correspondent} 
\hspace*{0pt}\hfill  \href{https://www.prisonmathproject.org/}{Prison Mathematics Project}, Summer 2021-



%	{\begin{newitemize}
%	\item{Association for Computing Machinery}
%		{\begin{newitemize}
%			\item{Member, 2019}
%		\end{newitemize}}
%	\item{CWRU Hacker Society}
%		{\begin{newitemize}
%			\item{ Member, 2017-Current}
%		\end{newitemize}}
%	
%	
%	
%	\end{newitemize}}
	}
\end{document}
